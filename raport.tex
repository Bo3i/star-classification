\documentclass{article}
\usepackage{polski}
\usepackage{graphicx}
\usepackage{float}
\usepackage{makeidx}

\title{Klasyfikacja ciał niebieskich}
\author{Agata Mironowa\and Palina Sauranskaya \and Jakub Kindlik }
\date{April 2023}
\begin{document}
\maketitle
\tableofcontents

\section{Streszcznie}
to na koniec
\section{Słowa kluczowe}
Klasyfikacja, Drzewa decyzyjne, Ciała niebieskie, ...
\section{Wprowadzenie}
\section{Przedmiot badań}
Celem projektu jest stworzenie hybrydowego modelu klasyfikującego ciała niebieskie na podstawie ich parametrów fizycznych.
W tym celu zostaną wykorzystane trzy metody klasyfikacji: X, drzewa decyzyjne oraz X.
W pierwszej części projektu zostaną przeprowadzone badania na rzeczywistych danych z bazy Sloan Digital Sky Survey
udostępnionej do domeny publicznej, a następnie na sztucznych danych.
W ostatniej części projektu zostanie stworzony model hybrydowy, który będzie łączył w sobie wszystkie trzy metody klasyfikacji.
\section{Dane}
Dane pochodzą ze strony Kaggle.com i są udostępnione do domeny publicznej. Zbiór danych został udostępniony przez
użytkownika FEDESORIANO i jest zbiorem danych z bazy Sloan Digital Sky Survey udostępnionym do domeny publicznej.
Zbiór danych zawiera informacje o 10 tysiącach ciał niebieskich.
\subsection{Opis danych}
Każde ciało niebieskie jest opisane przez 17 parametrów fizycznych.
\begin{enumerate}
    \item obj\_ID = Unikalny identyfikator obiektu w katalogu obrazów używanym przez CAS.
    \item alpha = Kąt rektascensji (w epoce J2000).
    \item delta = Kąt deklinacji (w epoce J2000).
    \item u = Filtr ultrawioletowy w systemie fotometrycznym.
    \item g = Filtr zielony w systemie fotometrycznym.
    \item r = Filtr czerwony w systemie fotometrycznym.
    \item i = Filtr bliskiej podczerwieni w systemie fotometrycznym.
    \item z = Filtr podczerwieni w systemie fotometrycznym.
    \item run\_ID = Numer skanu używany do identyfikacji konkretnego przebiegu.
    \item rereun\_ID = Numer powtórzenia, określający sposób przetworzenia obrazu.
    \item cam\_col = Kolumna kamery, służąca do identyfikacji linii skanu w ramach przebiegu.
    \item field\_ID = Numer pola, służący do identyfikacji każdego pola.
    \item spec\_obj\_ID = Unikalny identyfikator używany dla obiektów spektroskopowych (oznacza to, że dwie różne obserwacje o tym samym spec\_obj\_ID muszą mieć taką samą klasę wynikową).
    \item class = Klasa obiektu (galaktyka, gwiazda lub kwazar).
    \item redshift = Wartość przesunięcia ku czerwieni oparta na wzroście długości fali.
    \item plate = ID płyty, identyfikujące każdą płytę w SDSS.
    \item MJD = Zmodyfikowana data juliańska, używana do wskazania momentu pobrania danych SDSS.
    \item fiber\_ID = ID włókna, które skierowało światło na płaszczyznę ogniskową w każdej obserwacji.
    \end{enumerate}
\subsection{Przygotowanie danych}
Dane zostały pobrane ze strony Kaggle.com w formacie csv. Następnie zostały wczytane do pliku Jupyter Notebook \"
Prepare\_sets\". W tym pliku dane zostały podzielone na zbiór treningowy i testowy w proporcji 80\% do 20\%.
\subsection{Wstępna analiza danych}

\section{Opis metod}
\subsection{Metoda 1}
\subsection{Metoda 2}
\subsection{Metoda 3}
\subsection{Model hybrydowy}
\section{Rezultaty}
\section{Użycie modeli na sztucznych danych}
\section{Wnioski}
to na koniec
\section{Bibliografia}
to na koniec

\end{document}
